% Chapter Template

\chapter{Describing cells} % Main chapter title

\label{Chapter6} % Change X to a consecutive number; for referencing this chapter elsewhere, use \ref{ChapterX}


Once cells have been segmented and their fluorescence intensities classified, there are assigned with features that describe a human perception of the cells' properties. The interesting properties are summarized in the following section. 


%----------------------------------------------------------------------------------------
%	FEATURES
%----------------------------------------------------------------------------------------

\section{Interesting features}

In \cite{FoggiaBenchmarks2013}, Foggia et al. summarizes the description of interesting properties for each type of cells. Table \ref{tab:Desc} provides the description of every cell type. The problem with such descriptions is that they are quite unstructured and sometimes ubiquitous. For example, when speaking about organelles contained in a cell's body, they refer to dark organelles as \textit{granules} and bright organelles as \textit{speckles}. \\

In order to develop a method that will generate such descriptions automatically, those description should be structured first. Tables \ref{tab:Vpata} and \ref{tab:Vpatb} gives a description of each pattern in a more structured way. The appearance of patterns depends on the fluorescence intensity of an image. Several important visual patterns were identified. \\

The first property describes the shape of a cell. It can take two possible values - \textit{circular} and \textit{irregular}. The motivation for this feature is the observation that the irregular shape is very characteristic for the cytoplasmatic class, while all other classes have a circular shape. \\

The second property incorporates the knowledge about the mitotic cells found on the same image. The medical literature suggested that the mitotic cells, although there is no a general consensus about the importance of the mitotic cells, might be informative for discriminating between certain classes. The mitotic cells can appear in three different patterns - \textit{neutral} with no specific characteristics, \textit{bright center} of \textit{dark center}. \\

The following property describes the appearance of organelles in a cell's body, if any present. This property is very discriminatory towards nucleolar and centromere which have bright organelles, speckled which contain dark organelles while others don't have any organelles. \\

The number of organelles is also an informative property. The centromere class has significantly larger number of organelles contained in a body. The centromene, nucleolar and speckled class relatively small number of organelles expressed in the body, while the homogeneous and cytoplasmatic class don't have any organelles. \\

For certain classes, the texture could be quite informative. Although most of the classes have a relatively smooth texture, some classes have some very distinguishing properties. On images with the positive fluorescence intensity, the cytoplasmatic class has characteristic blob. Also, the coarse speckled class has a very specific texture that looks like a \textit{sea surface}. \\



\begin{table}
	\begin{center}
	\caption{Description of cell types}
	\label{tab:Desc}
	\begin{tabular}{|m{2.3cm}|m{2.1cm}|m{8cm}|}
		\hline
		\textbf{pattern type} & \textbf{example} & \textbf{description} \\
		\hline
		centromere & \includegraphics[width=2cm]{Figures/describing/centromere} & characterized by several 			discrete speckles ($ \approx 40-60$) distributed throughout the interphase nuclei and 		characteristically found in the condensed nuclear chromatin during mitosis as a bar of closely associated speckles. \\ \hline
		nucleolar & \vspace{5pt} \includegraphics[width=2cm]{Figures/describing/nucleolar} & characterized by clustered 			large granules in the nucleoli of interphase cells which tend towards homogeneity, with less than six granules per cell. \\ \hline
		homogeneous & \vspace{5pt} \includegraphics[width=2cm]{Figures/describing/homogeneous} & characterized by a 	diffuse staining of the interphase nuclei and staining of the chromatin of mitotic cells. \\ \hline
		
		fine speckled & \vspace{5pt} \includegraphics[width=2cm]{Figures/describing/fine_speckled} & characterized by a 			fine granular nuclear staining of the interphase cell nuclei \\ \hline
		
		coarse speckled & \vspace{5pt} \includegraphics[width=2cm]{Figures/describing/coarse_speckled} & characterized by a coarse granular nuclear staining of the interphase cell nuclei \\ \hline
		
		cytoplasmatic & \vspace{5pt} \includegraphics[width=2cm]{Figures/describing/cytoplasmatic} & characterized by a specific shape and large granule \\ \hline
	\end{tabular}
	\end{center}
\end{table}

\begin{table}
	\caption{Identified visual patterns - positive intensity}
	\label{tab:Vpata}
	\small
	\begin{tabular}{|m{2.2cm}|m{1.4cm}|m{1.5cm}|m{1.5cm}|m{1.4cm}|m{1.6cm}|m{1.4cm}|}
		\hline
		\textbf{pattern type} & \textbf{shape} & \textbf{mitotic cell} & \textbf{organelle type} & \textbf{organelle count} & \textbf{texture} & \textbf{speckles} \\ \hline
		centromere & circular & X & bright on dark & lots & sparkly  & yes \\ \hline
		nucleolar & circular & X & bright on dark & few & smooth & yes \\ \hline
		cytoplasmatic & irregular & X & dark on bright & none & blob (positive)  & no \\ \hline
		homogeneous & circular & X & neutral & none & smooth & no \\ \hline
		fine speckled & circular & X & neutral & none & smooth & no \\ \hline
		coarse speckled & circular & X & dark on bright & few & sparkly & yes \\ \hline
	\end{tabular}
	\normalsize
\end{table}

\begin{table}
	\caption{Identified visual patterns - intermediate intensity}
	\label{tab:Vpatb}
	\small
	\begin{tabular}{|m{2.2cm}|m{1.4cm}|m{1.5cm}|m{1.5cm}|m{1.4cm}|m{1.6cm}|m{1.4cm}|}
		\hline
		\textbf{pattern type} & \textbf{shape} & \textbf{mitotic cell} & \textbf{organelle type} & \textbf{organelle count} & \textbf{texture} & \textbf{speckles} \\ \hline
		centromere & circular & X & bright on dark & lots & sparkly  & yes \\ \hline
		nucleolar & circular & X & bright on dark & few & smooth & yes \\ \hline
		cytoplasmatic & irregular & X & neutral & none & smooth  & no \\ \hline
		homogeneous & circular & X & neutral & none & smooth & no \\ \hline
		fine speckled & circular & X & dark on bright & none & smooth & no \\ \hline
		coarse speckled & circular & X & dark on bright & few & sparkly & yes \\ \hline
	\end{tabular}
	\normalsize
\end{table}


%--------------------------------------------%
%                                            %
%             DEEP LEARNING                  %
%                                            %
%--------------------------------------------% 

\section{Deep learning}





%--------------------------------------------%
%                                            %
%           DEEP BELIEF NETWORKS             %
%                                            %
%--------------------------------------------%

\section{Deep belief networks}





%--------------------------------------------%
%                                            %
%               EVALUATION                   %
%                                            %
%--------------------------------------------%

\section{Evaluation}

