\documentclass[master=mai,english,openright,fleqn]{kulemt}
\setup{title={Classification of human epithelial cells' staining patterns},
  author={Sebastijan Dumancic},
  promotor={Prof.\,dr.\ Hendrik Blockeel},
  assessor={},
  assistant={Antoine Adam}}
  
% The following \setup may be removed entirely if no filing card is wanted%
%\setup{filingcard,
%  translatedtitle=,
%  udc=621.3,
%  shortabstract={Here comes a very short abstract, containing no more than 500
%    words. \LaTeX\ commands can be used here. Blank lines (or the command
%    \texttt{\string\pa r}) are not allowed!
%    \endgraf \lipsum[2]}}
% Uncomment the next line for generating the cover page
%\setup{coverpageonly}
% Uncomment the next \setup to generate only the first pages (e.g., if you
% are a Word user.
%\setup{frontpagesonly}

% Choose the main text font (e.g., Latin Modern)
\setup{font=palatino}

\chapterstyle{veelo}

% If you want to include other LaTeX packages, do it here. 
%\usepackage[utf8]{inputenc}
\usepackage{algpseudocode}
\usepackage{float}
\usepackage{algorithm}
\usepackage{amsmath}
\usepackage{tabulary}
\usepackage{ wasysym }
\usepackage{array}
\usepackage{graphicx}
\usepackage{multirow}
\usepackage{xcolor,colortbl}

\newcommand\MyBox[1]{
  \fbox{\lower0.75cm
    \vbox to 1.7cm{\vfil
      \hbox to 1.7cm{\hfil\parbox{1.0cm}{ #1}\hfil}
      \vfil}%
  }%
}

% Finally the hyperref package is used for pdf files.
% This can be commented out for printed versions.
\usepackage[pdfusetitle,colorlinks,plainpages=false]{hyperref}

%%%%%%%
% The lipsum package is used to generate random text.
% You never need this in a real master thesis text!
\IfFileExists{lipsum.sty}%
 {\usepackage{lipsum}\setlipsumdefault{11-13}}%
 {\newcommand{\lipsum}[1][11-13]{\par And some text: lipsum ##1.\par}}
%%%%%%%

%\includeonly{chap-n}
\begin{document}

\begin{preface}

	I would like to thank my supervisors, Prof. Hendrik Blockeel and Antoine Adam, for all the advices and support that made this Thesis to happen. I would also like to thank prof. Damir Sersic and prof. Jan Snajder from University of Zagreb, Croatia for lighting the initial spark about this area in me. \\
	
	My stay in Leuven as an Erasmus student wouldn't be possible without the support from my family. A huge gratitude  goes to them. Finally, this Erasmus year was a wonderful experience due to great people surrounding me. I am thankful for all of those moments with them.
  %I would like to thank everybody who kept me busy the last year,
  %especially my promotor and my assistants. I would also like to thank the
  %jury for reading the text. My sincere gratitude also goes to my wive and
  %the rest of my family.
  %\lipsum[1]
\end{preface}

\tableofcontents*

\begin{abstract}
  %The \texttt{abstract} environment contains a more extensive overview of
  %the work. But it should be limited to one page.

  This thesis proposes a solution for computer aided assistance in commonly used diagnostics in autoimmune diseases. The solution closely follows the procedure medical experts suggest. It covers the segmentation task,  the fluorescence intensity level classification and staining pattern classification. The strongest contribution of this thesis is the interpretable approach to prediction models that help medical experts. \\

The work on the segmentation part was motivated by two problems found in previous approaches. Due to the colouring procedure of fluorescence imaging, cells exhibit different properties which makes them hard to detect, especially in noisy images. By finding the background instead of cells directly, that problem was successfully solved with great improvement. Still, a lot of cells overlap.  By using the information about the background, morphological snakes were introduced. Although not always capable of separating overlapping cells, the method demonstrated very promising results. The problematic case were the patterns with bright organelles where those organelles were segmented. \\

The cells were then characterized with \textit{visual concepts} describing their appearance. Deep belief networks were used to map images, or pixel features to \textit{symbolic} ones. Deep approaches to machine learning demonstrated the great potential in symbolic feature recognition. Surprisingly, convolutional components seem not to improve the recognition, but the full potential of this approach is yet to be investigated. \\

The symbolic representations learned in the previous step are then used to mine rules that describe the patterns. Four commonly used rule mining algorithms were compared and proven to perform as accurately as state-of-the-art methods, but offer explanations about their decisions. I believe such an approach is needed when computers assist in making a diagnosis, based on images or other diagnostic tests.
\end{abstract}

% A list of figures and tables is optional
%\listoffigures
%\listoftables
% If you only have a few figures and tables you can use the following instead
\listoffiguresandtables
% The list of symbols is also optional.
% This list must be created manually, e.g., as follows:
\chapter{List of Abbreviations and Symbols}
\section*{Abbreviations}
\begin{flushleft}
  \renewcommand{\arraystretch}{1.1}
  \begin{tabularx}{\textwidth}{@{}p{12mm}X@{}}
    CAD & \textbf{C}omputer \textbf{A}ided \textbf{D}iagnostics \\
    IIF & \textbf{Indirect} \textbf{i}mmuno\textbf{f}lourescence \\
    ILP & \textbf{I}nductive \textbf{L}ogic \textbf{P}rogramming \\
    ANA & \textbf{A}nti-\textbf{n}uclear \textbf{a}ntiody \\
    SVM &  \textbf{S}upport \textbf{V}ector \textbf{M}achine \\
    ACWE & \textbf{A}ctive \textbf{C}ontours \textbf{W}ithout \textbf{E}dges \\
    PDE & \textbf{P}artial  \textbf{d}ifferential \textbf{e}quation \\
    RBM & \textbf{R}estricted \textbf{B}oltzmann \textbf{M}achine \\
    FOIL & \textbf{F}irst \textbf{O}rder \textbf{I}nductive \textbf{L}earner \\
    MDL & \textbf{M}inimum \textbf{D}escription \textbf{L}earner \\
  \end{tabularx}
\end{flushleft}
\section*{Symbols}
\begin{flushleft}
  \renewcommand{\arraystretch}{1.1}
  \begin{tabularx}{\textwidth}{@{}p{12mm}X@{}}
    $\mu$ & a mean of a probabilistic distribution \\
    $\Sigma$ & covariance matrix of a probabilistic distribution
  \end{tabularx}
\end{flushleft}

% Now comes the main text
\mainmatter



\chapter{Introduction} 

\label{Chapter1} 


%----------------------------------------------------------------------------------------
%	GENERAL INTRODUCTION
%----------------------------------------------------------------------------------------

In the last couple of decades, computers  have found a number of applications in biology and medicine. We may even say they have become an essential tool in revealing the questions of life. A significant role in those problems was played by machine learning, a branch of artificial intelligence concerned with the construction of models capable of learning from data . Probably the most inspiring example comes from the field of bioinformatics where scientists have used the methods from  statistics and artificial intelligence to sequence the human genome for the first time. If that was one of the first results, then, what can we expect from the future? \\

Besides the genomic data which is represented as a sequence of nucleotides, a great amount of  biological data can be aquired by different imaging methods such as microscopy, PET or CT imaging and others. That is were the medical imaging and bioimage analysis fields come from. The field of bioimage analysis studies the biological problems by examining an image, or image sequence of a process of interest, while the medical image analysis is concerned with developing of methods that will help in a medical diagnosis process based on imaging. \\

The above mentioned field of medical image analysis overlaps with a field of computer aided diagnosis (CAD) which collects a broader range of methods used as an assistance to the doctors. An application area of this Thesis would fit the best in that field. This thesis will take a direction in a specific task of assistance to an autoimmune disease diagnosis, with a special emphasis to human interpretable models. During the recent few years, this specific problem has been solved very efficiently. If the problem has been solved, why is this thesis taking another look at it? \\

The goal of the Thesis is to provide a  CAD method capable of making a decision based on a microscopic image of HEp-2 cell in a human intepretable way. To demonstrate the importance of intepretable model in CAD systems, I consider the following scenario.

%----------------------------------
%   MOTIVATIONAL SCENARIO
%----------------------------------

\section{Motivational Scenario}

Imagine a following scenario. \textit{Peter} is a doctor, an immunologist. Being an immunologist, his job is to find out which autoimmune disease a patient has. As that is an extremly hard task, Jules uses  computer assistance - an artifical intelligence program named \textit{Hal}. Hal's role is to confirm Peter's diagnosis, or to provide an additional insight if the decision is hard to make. \\

We are interested in the situation where Peter and Hal have made a conflicted decision. There are two cases representing the insight Hal can provide. Each case reflects Hal's interior structure : a \textbf{black-box} model\footnote{By black-box model we assume a model for which we can only observe it's output, not a decision process} or an \textbf{ interpretable} one. If Hal is the black-box model, it is not much of an assistance, as Hal can't elaborate it's decision. We can only agree we disagree. In this case, Hal is not much of a help. \\

On the other hand, if Hal is an interpretable model, it can elaborate it's decision and help Peter. If a wrong decision was made by Hal, Peter can observe it's model parameters, correct the wrong one(s), query again and see if new result supports it's decision. If a wrong decision was made by Peter, Peter can observe Hal's model, find the parameters he might have overseen and correct his diagnosis. \\

The example clearly demonstrates one  thing -- the importance of interpretable models for CAD systems. Computers have proven their ability of inferring from a large amount of data, usually outperforming humans, but in specific situations, a good and accurate model is not enough. We also need to interpret results if we are going to use them.

\section{Detecting the autoimmune diseases}

As mentioned already, this Thesis will address the specific topic of autoimmune diseases classification. The human immune system creates antibodies to fight against infections whereas antinuclear antibodies affect healthy tissues. The Antinuclear Autoantibodies test (ANA) is widely used to determine whether the immune system is developing antibodies or not. Indirect immunofluorescence (IIF ) with HEp-2 cells is the recommended method to diagnose the presence of antinuclear autoantibodies in patient serum. \\

IIF method consists of four consecutive steps:
\begin{itemize}
	\item cell segmentation
 	\item intensity level classification
 	\item mitosis detection
 	\item staining pattern classification
\end{itemize}

The final step usually classifies cell images into one of following patterns: homoge
neous, speckled, nucleolar, cytoplasmic, centromere (see figure \ref{fig:CellExamples}). Some variations may be introduced in different datasets due to large number of possible patterns. Those staining patterns further have a clinical associations with a specific autoimmune disease such as Scleroderma, malignant tumor, Lupus nephritis and others. \\

\begin{figure}[htbp]
	\centering
	\includegraphics[scale=0.7]{Figures/introduction/cell_examples}
	\rule{35em}{0.5pt}
	\caption[Cell examples]{HEp-2 cell patterns}
	\label{fig:CellExamples}

\end{figure}

Although IIF posses qualities such as high sensitivity and a large number of antigens
that can be detected, it suffers from numerous shortcomings. The most important ones
are liability to subjectivity and time and labour consuming. \\

In order to avoid any kind of subjectivity , there is a great need for standardization and formalization of the mentioned procedure. Addressing this problem calls for CAD techniques which combines methods from machine learning and image analysis.

\section{Problem statement}

The content of the thesis proposes solutions for three major steps in the IFF process. \\

First of all, in order to find out a  pattern, cells should be isolated from an image aquired by \textit{IFF}. As it is going to be explained (see chapter \ref{Chapter3}), although the problem of cell segmentation has been studied for more than 50 years, segmenting HEp-2 cells still suffers from certain problems, mostly due to different green flourescent protein absorbtion accross the cells. This thesis proposes a method based on a region growing algorithm that demonstrates an encouraging result for overcoming mentioned problems. \\

Once we have isolated the cells, the next step is to determine the fluorescent intensity of an image. The fluorescent intensity level describes a subjective measure of \textit{clarity} of an image. This parameter is treated as a feature later in the process, but due to a sake of procedure, this step is developed separately. 

The final step presents a novel approach to the staining pattern classification. The current state of the art solution solves the problem very successfully, but acts like  black-box solutions not providing any explanation for the decision. This thesis tends to develop a method based on the human interpretable representation and reasoning. As IF-THEN rules are the most natural way of representing human knowledge,  the thesis will follow a rule mining approach, such as Inductive Logic Programming (ILP). \\
% Chapter Template

\chapter{Background} 

\label{Chapter2} % Change X to a consecutive number; for referencing this chapter elsewhere, use \ref{ChapterX}

%\lhead{Chapter 2. \emph{Understanding the domain}} % Change X to a consecutive number; this is for the header on each page - perhaps a shortened title

%----------------------------------------------------------------------------------------
%	SECTION 1
%----------------------------------------------------------------------------------------

\section{Understanding the domain}

\subsection{Immune system and antibodies }

The immune system is the central part of the human body responsible for protection against infections. In order to function properly, the immune system has to detect a wide range of threats, and at the same time distinguish them from  healthy tissue. The main weapon the immune system can use are antibodies.   \\

The antibody is a protein complex produced by \textit{B cells} \footnote{ a subgroup of white blood cells, a viral part of the immune system} that initiates an immune response against a target antigen \footnote{Foreign substance that, when introduced into the body, is capable of stimulating an immune response}. Their primal role is to recognize the unique part of the foreign target and  protect the body from infections. The basic organization of the antibody includes two functional domains that, together, resemble the letter Y (Figure \ref{fig:AntibodyIllustration}, left). The \textit{Fab}  part makes up the arms of the Y, and it contains the antigen-binding site - the region responsible for antigen binding. The \textit{Fc} part comprises the tail of the Y and effects other cells, proteins and antibodies.  \\

This unique structure allows detection of antigens, in direct or indirect manner. By direct detection we assume detection using a single fluorophore-labeled antibody, and by indirect detection we assume detection through binding of a fluorophore-labeled secondary antibody raised against the Fc part of an unlabeled primary antibody (as illustrated in Figure \ref{fig:AntibodyIllustration}, right). This system is versatile and cost-effective because few labeled antibodies are required to detect many possible primary antibodies. \\


\begin{figure}[htbp]
	\centering
	\includegraphics[scale=0.5]{Figures/Domain/antibodies}
	\rule{35em}{0.5pt}
	\caption[Illustration of antibody structure]{Illustration of antibody structure (from \cite{OdellCook2013})}
	\label{fig:AntibodyIllustration}

\end{figure}

The immune system can sometimes suffer from different disorders. A disorder of special importance to this Thesis is  \textit{autoimmunity}. Autoimmunity results in the disability of the immune system to recognize an organism's healthy tissue, and therefore attacks normal tissues as if they were foreign organisms. In the case of autoimmunity, antibodies are called antinuclear antibodies. We can observe those antibodies by using indirect immunofluorescence. \\

\subsection{Indirect Immunofluorescence}

As it was already mentioned, the indirect detection is the main focus of the thesis. Indirect immunofluorescence is a diagnostic methodology based on image analysis that reveals the presence of autoimmune diseases by searching for antibodies in the patient serum. Indirect immunofluorescence is a two-step technique, in which a primary, unlabeled antibody binds to the target, after which a fluorophore-labeled\footnote{fluorophore-labeling is a method to color the antibodies so they can be observed under the microscope} second antibody is used to detect the first antibody (figure \ref{fig:AntibodyIllustration}, right). Indirect immunofluorscence is more sensitive than a direct one because more than one secondary antibody can bind to each primary antibody, which amplifies the fluorescence signal. \\

As a result of its effectiveness,there has been a growing demand for diagnostic tests for systemic autoimmune diseases. Unfortunately,  IIF still remains a subjective method that depends too heavily on the experience and expertise of the physician. The main reasons causing the problems are:
\begin{itemize}
	\item the lack of quantitative information supplied to physicians
	\item varieties of reading systems and optics
	\item the photo-bleaching effect caused by a light source irradiating the cells over a short period of time
	\item the low reproducibility of the diagnostic protocol.
\end{itemize}

\subsection{Putting it all together}

The focus of this thesis is on the Antinuclear antibodies  test (ANA), which plays the main role in the serological\footnote{Further explanation} diagnosis of autoimmune disease. ANAs are directed against a variety of antigens and can be detected in patient serum through laboratory tests. IIF uses the human epithelial (HEp-2) substrate, which bonds with serum antibodies forming a molecular complex. This complex then reacts with human immunoglobulin \footnote{Immunoglobulin is a specific type of antibody created by plasma cells} and becomes visible under a fluorescence microscope which reveals the antigen-antibody reaction. \\

The procedure of ANA starts with  fluorescence intensity classification, a segmentation step is not a part of ANA procedure. The Center for Disease Control and Prevention in Atlanta,USA have published  guidelines \cite{nakamura1996quality} for scoring the intensity. The score ranges from 0 to 4+ as follows:
\begin{itemize}
	\item 4+ : brilliant green (maximal fluorescence)
	\item 3+ : less brilliant green fluorescence
	\item 2+ : defined pattern but diminished fluorescence
	\item 1+ : very subdued fluorescence
	\item 0 : negative.
\end{itemize}

Although the guidelines provide  very detailed instructions, in \cite{Rigon2007} Rigon et al. analyzed the variability between a set of physician's fluorescence intensity  classifications. Their work has shown a big variance of classifications made by physicians on the same dataset, so they suggested to classify the fluorescence intensity into three classes, namely negative, intermediate and positive. This work follows the protocol. \\

The final step consists of staining pattern recognition. As shown in figure \ref{fig:CellExamples}, there are several patterns that may be observed. \cite{FoggiaBenchmarks2013} and \cite{Perner02miningknowledge} provide a description of all staining patterns which is a valuable input taking into consideration a human interpretable perspective of this step. A summary is presented here:
\begin{itemize}

	\item \textbf{Centromere}: characterized by several discrete speckles ( $\sim$ 40 - 60) distributed throughout the interphase\footnote{The interphase is the nonmitotic phase of the cell cycle in which the cell spends the majority of its time and performs the majority of its purposes} nuclei and characteristically found in the condensed nuclear chromatin during mitosis as a bar of closely associated speckles.
	
	\item \textbf{Nucleolar}: characterized by clustered large granules in
the nucleoli of interphase cells which tend towards homogeneity, with less than six granules per cell.

	\item \textbf{Homogeneous}: characterized by a diffuse staining of the interphase nuclei and staining of the chromatin of mitotic cells.
	
	\item \textbf{Fine Speckled}: characterized by a fine granular nuclear staining of the interphase cell nuclei.
	
	\item \textbf{Coarse Speckled}: characterized by a coarse granular nuclear staining of the interphase cell nuclei.
	
	\item \textbf{Cytoplasmatic}: characterized by a highly irregular shape and large granule in the nucleoli

\end{itemize}


%------------------------------------------%
%                                          %
%           MACHINE LEARNING               %
%                                          %
%------------------------------------------%

\section{Machine learning}
% Chapter Template

\chapter{Literature overview} 

\label{Chapter3} % Change X to a consecutive number; for referencing this chapter elsewhere, use \ref{ChapterX}

%\lhead{Chapter 3. \emph{Literature overview}} % Change X to a consecutive number; this is for the header on each page - perhaps a shortened title

%----------------------------------------------------------------------------------------
%	SEGMENTATION
%----------------------------------------------------------------------------------------

\section{Cell segmentation}

Several studies have been proposed to classify autoantibody fluorescence patterns by using an automatic thresholding method, i.e.  Otsu's method, to segment the cells. The  thresholding method can choose the threshold to minimize the intraclass variance of the black and white pixels automatically. Due to the variety of ANA patterns,  Otsu's algorithm always failed to segment cells of speckled and nucleolar patterns, such as cases of a very blury image of low intensity. Figure \ref{fig:BadSegment} shows the over-segmentation results by using Otsu's algorithm. Additional challenge for the segmentation here is to separate overlapping cells which are quite common in the process.

\begin{figure}[htbp]
	\centering
	\includegraphics[scale=0.4]{Figures/introduction/badsegmentation}
	\rule{35em}{0.5pt}
	\caption[Bad segmentation example]{Segmentation results of the Otsu method (from \cite{Huang2008})}
	\label{fig:BadSegment}

\end{figure}


In \cite{Huang2008}, Huang et all. present an adaptive edged-based segmentation method for automatically detecting outlines of fluorescence cells in IIF images. Their approach is based on specific properties of the images regarding the pattern class. They have divided the images in two groups : sparse region and mass region cells. The mass region cells are those ones which have a \textit{compact} appearance, that look like a smooth object, while the sparse region cells are those ones for which we can detect multiple object in a cell. The approach trains a classifier to classify each image in the groups and applies different segmentation procedure for each group. In the case of the mass region cells, the cells are segmented using Otsu segmentation method, while in the case of the sparse region cell segmentation is performed by an edge detection. Their approach resulted in better segmentation results, but approximately 10\% of the cells remained undetected. \\

In \cite{HuangWatershed}, same authors further improve their method by incorporating watershed segmentation. The second approach also includes segmentation in two stages, depending on defined criteria. After a segmentation step performed by the watershed, the approach merges parts located relatively close and eliminates parts not large enough to represent a cell. If the retrieved number of regions doesn't safisfy the defined criteria, the segmentation step is performed again with different parameter settings determinated by Otsu's thresholding. \\

All forementioned approaches report similar shortcomings : approximately 10\% of cells remained undetected and the inability to separate overlapping cells.  The focus of the segmentation part of the Thesis will be on overcoming those problems. \\



%-----------------------------------------------------------------------------------------
%   INTENSITY LEVEL CLASSIFICATION
%-----------------------------------------------------------------------------------------

\section{Intesity level classification}

The following step, the intensity level classification, hasn't attracted a lot of scientific research, but has demonstrated remarkable results so far. \\

In \cite{SodaIntensity2006},  authors propose a system based on  \textit{Multi-Layer Perceptrons} and a \textit{Radial Basis Network} for the intensity classification step. That system, which makes use of features inspired from medical practice, shows error rates  up to 1\%, but it uses a reject option and it does not cast a result in about 50\% of cases. In \cite{SodaIntensity2009} the authors further refine their system. They train three experts, one specialized for each class, with a different set of features. They threat the classifiers similar to the \textit{one-vs-all} approach, so the final decision is made by a classifier most certain in it's decision. The authors report a success rate of 92,6\% accuracy. \\


%-----------------------------------------------------------------------------------------
%   STAINING PATTERN CLASSIFICATION
%-----------------------------------------------------------------------------------------

\section{Staining pattern classification}

As this problem was emphasized on the \textit{International Conference on Pattern Recognition 2012} as a contest, this step has been well researched and several very successful methods have been proposed. In \cite{FoggiaBenchmarks2013}, Foggia et al. provides a detailed overview of the methods submitted for the contest. The three most successful ones are presented here.  \\

In \cite{Kuan2012}, Kuan presents a method based on four texture descriptors: a rotation invariant form of local binary patterns (LBP) with multi-scale analysis,discrete cosine transformation, the mean values and standard variances of 2-D Gabor wavelets, and some global appearance based statistical features. A multiclass SVM was trained on each class of the four feature sets. The SVMs are then integrated in one classifier by using the AdaBoost.M1 algorithm. \\

In \cite{Nosaka2012}, Nosaka presents a similar approach on an extension of LBP, namely CoALBP \cite{Nosaka2011}. The advantage of this method is that the method can observe not only locals LBP, but also the spatial relations among adjacent LBP. The classifier is a linear SVM trained on an extended dataset including the rotated patterns of the original images. \\

Xianfeng et al. proposed a system based on MR8 method \cite{Varma2005} to extract statistical intensity features. The method calculates filter responses locally on the image, and then trains a global texton dictionary using $K$-means clustering. In that way, each image is represented by the frequency histogram of textons. The decision is made by a $k$-NN classifier. \\

Although there are many more papers in existance covering this problem, they are not presented here due to different, and not so rigorous, evaluation. Most of the early work was done on private datasets not available to public, which makes them not comparable to the new research results. Other papers with  a more recent date do not follow the evaluation procedure, so it is very hard to compare their efficiency with those ones in the overview. \\

More recently, a new overview of the method was presented by Agrawal et al. in \cite{Agrawal2013}. The committee of \textit{ICPR'13} has released a new, much bigger dataset for the same problem. The authors experimented with the most commonly used features in the previous contest - statistical features, histograms of oriented gradients, shape and size descriptors and texture descriptors. They have chosen the most typical representatives of classifiers, namely Naive Bayes, k-NN, SVM and Random forest. The SVM with Law's textural representation significantly  outperformed other classifiers and feature representations.



% Chapter Template

\chapter{Planing} % Main chapter title

\label{Chapter4} % Change X to a consecutive number; for referencing this chapter elsewhere, use \ref{ChapterX}

\lhead{Chapter 4. \emph{Planing}} % Change X to a consecutive number; this is for the header on each page - perhaps a shortened title

%----------------------------------------------------------------------------------------
%	SECTION 1
%----------------------------------------------------------------------------------------

In the segmentation part, the integration of a shape prior is to be done. I am still examining the literature due to different integration procedures. Most of my focus in January and February will be on this part.

For the fluorescent intensity step, I will try to improve the accuracy of so far used approach based on the approximation of image histogram by Gaussian mixture models.

Finally, I am still searching for a method that will allow me to capture a shape and texture representation of an object in a human interpretable way. After the exam period, I will mainly focused on this part.

% Chapter Template

\chapter{Fluorescence intensity classification} % Main chapter title

\label{Chapter5} % Change X to a consecutive number; for referencing this chapter elsewhere, use \ref{ChapterX}


%----------------------------------------------------------------------------------------
%	INTRODUCTION
%----------------------------------------------------------------------------------------

The fluorescence intensity is a parameter that describes the \textit{clarity} of cells in an image. It is a subjective parameter which, unfortunately, doesn't have a strong theoretical explanation. The fluorescence intensity is described with three values - positive, intermediate and negative. The positive value defines images in which cells are perfectly separated from the background, while the negative value defines images in which cells  cannot be identified. The intermediate value covers images that are nor positive nor negative. This parameter is used as a symbolic feature later in the process, but for the sake of \textit{standard procedure} it is estimated separately. \\


In \cite{Rigon2007} Rigon studied the variability of decisions made by doctors and showed that it is difficult to achieve a consensus about unique determination of the fluorescence intensity value. The lack of the underlying model is making this problem hard to formulate. The intuition behind the suggested approach is an assumption that the intensity level could be observed in the histogram of an image. The fluorescence intensity should correspond to the difference of a region describing the background and a region describing cells. Following the intuition, the image histogram is approximated with the Gaussian mixture model. \\

The idea is to approximate the histogram with a mixture of two Gaussian functions - one representing the background and second one to model the cells. The intuition is that images with positive intensity should have Gaussians with higher means and more further apart. 


%-----------------------------------------------------------------------------------------
%     Classifying intensity
%-----------------------------------------------------------------------------------------

\section{Classifying intensity}

Once the histogram has been approximated with two Gaussian functions, the estimated means and variances has been taken as features for the classification. A \textit{support vector machine} with \textit{radial basis functions} as kernel function has been trained for the task. Evaluation is performed using a 10-fold cross validation. \\



%--------------------------------------------------%
%                                                  %
%               SVM                                %
%                                                  %
%--------------------------------------------------%


\subsection{Support Vector Machine}


The Support vector machine is a very popular machine learning technique. It is a representative of a more general class of \textit{kernel methods}. The most interesting property of the support vector machine is that it tries to find the \textit{optimal hyperplane} that separates classes. Consider a two-class case. The sample is $\mathcal{D} = \{ \mathbf{x}^{(i)}, y^{(i)} \}_{i=1}^N$ where $y^{(i)} = 1$ if $\mathbf{x}^{(i)} \in \mathcal{C}_1$ and $y^{(i)} = -1$ if $\mathbf{x}^{(i)} \in \mathcal{C}_2$. We want to find a margin $\mathbf{w}$ so that

\begin{equation*}
	\mathbf{w}^T\mathbf{x}^{(i)} + w_o \geq +1 \quad \mathtt{for} \quad y^{(i)} = 1
\end{equation*}

\begin{equation*}
	\mathbf{w}^T\mathbf{x}^{(i)} + w_o \leq -1 \quad \mathtt{for} \quad y^{(i)} = -1
\end{equation*}

or simplified 

\begin{equation*}
	y^{(i)}(\mathbf{w}^T\mathbf{x}^{(i)} + w_0) \geq 1.
\end{equation*}

The support vector machine extends the \textit{basic} hyperplane requirement - setting the instances on the right side of the hyperplane - by trying to maximize the margin - the distance from the hyperplane to the instances closest to it. So, the \textit{optimal separating hyperplane} is the one that maximizes the margin. \\

The distance of $\mathbf{x}^{(i)}$ to the hyperplane equals to 

\begin{equation}
	\frac{ | \mathbf{w}^T\mathbf{x}^{(i)} - w_0 | }{ \Vert \mathbf{w} \Vert}.
\end{equation}

Finding the maximal margin can be expressed as

\begin{equation}
	\frac{y^{(i)}(\mathbf{w}^T\mathbf{x} - w_0)}{\Vert \mathbf{w} \Vert} \geq \rho, \forall i,
\end{equation}

i.e., we want the margin to be at least $\rho$. As there are infinitely many solutions that can be obtained by scaling $\mathbf{w}$, we fix the $\rho \Vert \mathbf{w} \Vert = 1$ and minimize $\Vert \mathbf{w} \Vert$ to maximize the margin. The problem can now be written in a form of quadratic optimization problem

\begin{align*}
	\min_{\mathbf{w}, w_0} & \quad \frac{1}{2} \Vert \mathbf{w} \Vert \\
	\mathtt{subject \ to} & \quad  y^{(i)} \left ( \mathbf{w}^T\mathbf{x} + w_0  \right ) \geq 1.
\end{align*}

Now there will be instances that are $\frac{1}{\Vert \mathbf{w} \Vert}$ away from the hyperplane and the total margin equals to $\frac{2}{\Vert \mathbf{w} \Vert}$. \\

If the problem is not linearly separable, one trick is to map the problem to a new space, with more dimension that the original space, by using nonlinear basis functions. It is generally the case that higher dimensional representation is easier to separate. One of those mapping functions are \textit{radial basis functions}. 


%---------------------------------------%
%                                       %
%                RBF                    %
%                                       %
%---------------------------------------%
 
\subsection{Radial basis functions}

Radial basis functions are an instance of a \textit{local representation}, where for a given input, only a few factors are \textit{active}. It is in a way a partition of space so that \textit{locally tuned} partitions are selective to only certain inputs. \\

With the concept of local partitioning we need to define a measure of similarity between an input $\mathbf{x}^{(i)}$ and local clusters $\boldsymbol \mu^1, \ldots, \boldsymbol \mu^n$. Radial basis functions are defined as

\begin{equation}
	\mathbf{r}(\mathbf{x}^{(i)}, \boldsymbol \mu^k) = exp \left (- \frac{\Vert \mathbf{x}^{(i)}  - \boldsymbol \mu^k \Vert}{2\sigma_k^2} \right ),
\end{equation} 

that is, it uses the Euclidean distance as a measure of similarity and Gaussian function as a response function. The response function express a property of having a maximum where $\mathbf{x}^{(i)} = \boldsymbol \mu^k$ and decreasing as they get less similar. 




%--------------------------------------%
%                                      %
%           EVALUATION                 %
%                                      %
%--------------------------------------%

\subsection{Results}

One issue that might occur is a bad estimation of the background or cell body as some cells takes a very small portion of an image or, on the other hand, takes almost a whole image when segmented as shown in image \ref{fig:Bad}. To see how this influences the problem, histograms are approximated with Gaussian functions in two ways. First, the histogram is as in Chapter \ref{Chapter4}, without any restrictions. Second, the background and cells part of an image are separated and each part is approximated with a Gaussian individually. \\

\begin{figure}
	\begin{minipage}[h]{0.32\linewidth}
		\begin{flushright}
			\includegraphics[height=2cm]{Figures/intensity/image1}
		\end{flushright}
	\end{minipage}
	\begin{minipage}[h]{0.32\linewidth}
		\centering
		\includegraphics[height=2cm]{Figures/intensity/image2}
	\end{minipage}
	\begin{minipage}[h]{0.32\linewidth}
		\begin{flushleft}
			\includegraphics[height=2cm]{Figures/intensity/image3}
		\end{flushleft}
	\end{minipage}
	\caption{Potential image that could obtain bad estimation}
	\label{fig:Bad}
\end{figure}

Table \ref{res:SepReg} summarizes the results obtained with separation of background and cell regions (a) and without (b). The accuracy of the proposed solution is 96.21 \% and 96.08 \% for separation and without it, respectively. As the results for both cases don't show any significant difference, the stability of the method is promising.  The best performance reported so far is 92,6 \% although the authors have used a private data set that is not available online, so it is hard to make a comparison.  So far, the results of this data set are not yet reported. \\

So far, the evaluation has been performed on  cell level - the intensity level is estimated on cell level. In practice, the intensity level is assigned on the image level, which means that every cell in an image is assigned to the same value of the parameter, regardless of the variations of a specific subset of cells. The intensity level of an image is then determined by averaging the assigned intensity levels of the cells on the image. With that approach, a 100 \% accuracy is achieved.




\begin{figure}
	\caption{Results of intensity classification}
	\label{res:SepReg}
	\begin{minipage}[h]{0.49\linewidth}
		\begin{center}
			\begin{tabular}{c c| c c}
				 & & \multicolumn{2}{c}{True} \\
			     & & \textbf{pos} & \textbf{int} \\
			    \hline
			    \multirow{2}{*}{\rotatebox[origin=c]{90}{Pred}} & \textbf{pos} & 804 & 31 \\
			    & \textbf{int} & 25 & 595
			\end{tabular} \\
		a)
		\end{center}
	\end{minipage}
	\begin{minipage}[h]{0.49\linewidth}
		\begin{center}
			\begin{tabular}{c c| c c}
				 & & \multicolumn{2}{c}{True} \\
			     & & \textbf{pos} & \textbf{int} \\
			    \hline
			    \multirow{2}{*}{\rotatebox[origin=c]{90}{Pred}} & \textbf{pos} & 800 & 36 \\
			    & \textbf{int} & 29 & 590
			\end{tabular} \\
			b)
		\end{center}
	\end{minipage}
\end{figure}


% Chapter Template

\chapter{Describing cells} % Main chapter title

\label{Chapter6} % Change X to a consecutive number; for referencing this chapter elsewhere, use \ref{ChapterX}


Once cells have been segmented and their fluorescence intensities classified, there are assigned with features that describe a human perception of the cells' properties. The interesting properties are summarized in the following section. 


%----------------------------------------------------------------------------------------
%	FEATURES
%----------------------------------------------------------------------------------------

\section{Interesting features}

In \cite{FoggiaBenchmarks2013}, Foggia et al. summarizes the description of interesting properties for each type of cells. Table \ref{tab:Desc} provides the description of every cell type. The problem with such descriptions is that they are quite unstructured and sometimes ubiquitous. For example, when speaking about organelles contained in a cell's body, they refer to dark organelles as \textit{granules} and bright organelles as \textit{speckles}. \\

In order to develop a method that will generate such descriptions automatically, those description should be structured first. Tables \ref{tab:Vpata} and \ref{tab:Vpatb} gives a description of each pattern in a more structured way. Several important visual patterns were identified. 



\begin{table}
	\begin{center}
	\caption{Description of cell types}
	\label{tab:Desc}
	
	\begin{tabular}{|m{2.3cm}|m{2.1cm}|m{8cm}|}
		\hline
		\textbf{pattern type} & \textbf{example} & \textbf{description} \\
		\hline
		centromere & \includegraphics[width=2cm]{Figures/describing/centromere} & characterized by several 			discrete speckles ($ \approx 40-60$) distributed throughout the interphase nuclei and 		characteristically found in the condensed nuclear chromatin during mitosis as a bar of closely associated speckles. \\ \hline
		nucleolar & \vspace{5pt} \includegraphics[width=2cm]{Figures/describing/nucleolar} & characterized by clustered 			large granules in the nucleoli of interphase cells which tend towards homogeneity, with less than six granules per cell. \\ \hline
		homogeneous & \vspace{5pt} \includegraphics[width=2cm]{Figures/describing/homogeneous} & characterized by a 	diffuse staining of the interphase nuclei and staining of the chromatin of mitotic cells. \\ \hline
		
		fine speckled & \vspace{5pt} \includegraphics[width=2cm]{Figures/describing/fine_speckled} & characterized by a 			fine granular nuclear staining of the interphase cell nuclei \\ \hline
		
		coarse speckled & \vspace{5pt} \includegraphics[width=2cm]{Figures/describing/coarse_speckled} & characterized by a coarse granular nuclear staining of the interphase cell nuclei \\ \hline
		
		cytoplasmatic & \vspace{5pt} \includegraphics[width=2cm]{Figures/describing/cytoplasmatic} & characterized by a specific shape and large granule \\ \hline
	\end{tabular}
	\end{center}
\end{table}

\begin{table}
	\caption{Identified visual patterns - positive intensity}
	\label{tab:Vpata}
	\begin{tabular}{|m{2.2cm}|m{1.7cm}|m{2cm}|m{2cm}|m{1.2cm}|m{2cm}|m{1.2cm}|}
		\hline
		\textbf{pattern type} & \textbf{shape} & \textbf{mitotic cell} & \textbf{organelle type} & \textbf{organelle count} & \textbf{texture} & \textbf{speckles} \\ \hline
		centromere & circular & X & bright on dark & lots & sparkly  & yes \\ \hline
		nucleolar & circular & X & bright on dark & few & smooth & yes \\ \hline
		cytoplasmatic & irregular & X & dark on bright & none & blob (positive)  & no \\ \hline
		homogeneous & circular & X & neutral & none & smooth & no \\ \hline
		fine speckled & circular & X & neutral & none & smooth & no \\ \hline
		coarse speckled & circular & X & dark on bright & few & sparkly & yes \\ \hline
	\end{tabular}
\end{table}

\begin{table}
	\caption{Identified visual patterns - intermediate intensity}
	\label{tab:Vpatb}
	\begin{tabular}{|m{2.2cm}|m{1.7cm}|m{2cm}|m{2cm}|m{1.2cm}|m{2cm}|m{1.2cm}|}
		\hline
		\textbf{pattern type} & \textbf{shape} & \textbf{mitotic cell} & \textbf{organelle type} & \textbf{organelle count} & \textbf{texture} & \textbf{speckles} \\ \hline
		centromere & circular & X & bright on dark & lots & sparkly  & yes \\ \hline
		nucleolar & circular & X & bright on dark & few & smooth & yes \\ \hline
		cytoplasmatic & irregular & X & neutral & none & smooth  & no \\ \hline
		homogeneous & circular & X & neutral & none & smooth & no \\ \hline
		fine speckled & circular & X & dark on bright & none & smooth & no \\ \hline
		coarse speckled & circular & X & dark on bright & few & sparkly & yes \\ \hline
	\end{tabular}
\end{table}


%--------------------------------------------%
%                                            %
%             DEEP LEARNING                  %
%                                            %
%--------------------------------------------% 

\section{Deep learning}





%--------------------------------------------%
%                                            %
%           DEEP BELIEF NETWORKS             %
%                                            %
%--------------------------------------------%

\section{Deep belief networks}





%--------------------------------------------%
%                                            %
%               EVALUATION                   %
%                                            %
%--------------------------------------------%

\section{Evaluation}


% Chapter Template

\chapter{Rule mining} % Main chapter title

\label{Chapter7} % Change X to a consecutive number; for referencing this chapter elsewhere, use \ref{ChapterX}


\texttt{IF-THEN} rules are the most natural way of expressing knowledge. They are easy to understand and quite expressive. Because of their nature to represent knowledge, mining knowledge in such form is of interest to this project. \\

The symbolic representation of cells obtained in the previous step is here used as an input features. Two popular approaches for mining such rules in a supervised way are \textit{Decision trees} and \textit{Inductive Logic Programming}.



%----------------------------------------
% DECISION TREE
%----------------------------------------

\section{Decision Trees}

Decision trees are one popular instance of machine learning algorithms due to their \textit{simplicity} and \textit{interpretability}. The decision tree approximates discrete functions and it is very robust to noisy data which makes it suitable for this application. It uses the \textit{tree} structure to represent and compress the data where each \textit{node} represents a test on attribute and each \textit{branch} stands for values of the tested attribute. \\

The decision tree performs a \textit{greedy search} in a search space and chooses an attribute to split data on. The chosen attribute then becomes the node in the tree, and values it can take become branches. As a measure of \textit{quality} of an attribute, \textit{information gain} is used. \\

Information gain compares the \textit{entropy} of the dataset before and after splitting. For a $K$ class case, entropy of data set $\mathcal{D}$ is defined as

\begin{equation}
	\mathtt{Entropy}(\mathcal{D}) = \sum_{k=1}^K p(\mathcal{C}_k)\mathtt{log}_2 p(\mathcal{C}_k)
\end{equation}

where $p(\mathcal{C_k})$ stands for a fraction of examples in data set $\mathcal{D}$ that belong to class $\mathcal{C}_k$. Informally, it measures the \textit{impurity} of data. Having entropy defined, information gain achieved by splitting on attribute $A$ is defined as

\begin{equation}
	\mathtt{InformationGain}(\mathcal{D}, A) = \mathtt{Entropy}(\mathcal{D}) - \sum_{v \in values(A)} \frac{\vert \mathcal{D}_v \vert}{\vert \mathcal{D} \vert}\mathtt{Entropy}(\mathcal{D}_v)
\end{equation} 

where $\mathcal{D}_v = \{ x \in \mathcal{D} | A(x) = v \}$ represents a subset of the original data set $\mathcal{D}$ with attribute $A$ equal to $v$. Informally, as entropy measures the \textit{impurity} of data, information gain measures how much of the \textit{impurity} has been lost by splitting the data. The partitioning continues until there are not attributes to split on, or the data contains only examples for the same class. Decision trees are summarized in Algorithm \ref{alg:DT}.


\begin{algorithm}
	\caption{Decision Tree learning}
	\label{alg:DT}
	\begin{algorithmic}[1]
		\Function{DecisionTree}{$\mathcal{D}$, \textbf{X}}
			\State $T \gets$ new tree
			\If{ all instances in $\mathcal{D}$ have same class $c$}
				\State $\mathtt{Label}(T) = c$
				\State return $T$
			\EndIf	
			\If{ \textbf{X} = \o}
				\State $\mathtt{Label}(T) = $ most common class in $\mathcal{D}$
				\State return $T$
			\EndIf 
			\State $X \gets$ attribute with highest information gain
			\State $\mathtt{Label}(T) = X$
			\For{ each $x$ in \textbf{X}}
				\State $\mathcal{D}_x \gets $ examples from $\mathcal{D}$ with $X = x$
				\If{$\mathcal{D}_x$ is empty}
					\State let $T_x$ be a new tree
					\State $\mathtt{Label}(T_x) \gets $ most common class in $\mathcal{D}_x$
				\Else
					\State $T_x \gets \text{ DecisionTree}(\mathcal{D}_x, \mathbf{X} - \{ X \})$
				\EndIf
				\State add a branch from $T$ to $T_x$
			\EndFor
			\State return $T$
		\EndFunction 
	\end{algorithmic}
\end{algorithm}

%----------------------------------------------------------------------------------------
%	ILP
%----------------------------------------------------------------------------------------

\section{Inductive logic programming}

Inductive logic programming (ILP) is an intersection between Machine learning and Logic programming where logic representation is used to induce knowledge. Inductive logic programming is concerned with finding a hypothesis $\mathcal{H}$ from a set of positive and negative examples $\mathcal{P}$ and $\mathcal{N}$. It is required that the hypothesis $\mathcal{H}$ covers all positive examples in $\mathcal{P}$ and none of the negative examples in $\mathcal{N}$. \\

ILP programs incorporate \textit{background knowledge} to induce rules from positive examples. It is convenient to view the background knowledge $\mathcal{B}$ as a logic program that is provided to the ILP system and fixed during the learning process. Under the presence of background knowledge, the hypothesis $\mathcal{H}$, together with the background theory $\mathcal{B}$, should cover all positive and none of the negative examples. \\

Many ILP systems have been developed over the years and three representative systems are described here.

\subsection{FOIL}

FOIL algorithm is an instance of \textit{Sequential covering} algorithms. Sequential covering employs the \textit{divide-and-conquer} principle and tries to learn one rule at time. Every induce rule should cover as many positive examples as possible, and as less negative examples as possible. \\

FOIL is summarized in Algorithm \ref{alg:FOIL}. FOIL performs a general-to-specific search where each candidate clause to a current rule is in one of the following forms:

\begin{itemize}
	\item $\mathcal{Q}(v_1,\ldots, v_n)$ where $\mathcal{Q}$ is a predicate from the set of all predicates and $v_i$ are either variables already present in the rule or new variables
	\item $\mathtt{Equal}(x_i, x_j)$ where $x_i \text{ and } x_j$ are variables already present in the rule
	\item the negation of either of the above mentioned forms.
\end{itemize}

Another important step in FOIL is the evaluation of candidate clauses. Let $R'$ be the rule created by adding candidate clause $L$ to rule $R$. Performance measure $\mathtt{FoilGain}(L,R)$ is defined as

\begin{equation}
	\mathtt{FoilGain}(L,R) = t \left ( \mathtt{log}_2 \frac{p_1}{p_1 + n_1} - \mathtt{log}_2 \frac{p_0}{p_0 + n_0} \right ),
\end{equation}

where $p_0$ stands for the number of positive bindings of rule $R$, $n_0$ is the number of negative bindings of $R$, while $p_1$ and $n_1$ represent the number of positive and negative bindings of $R'$. 

\begin{algorithm}
	\caption{FirstOrderInductiveLearner}
	\label{alg:FOIL}
	\begin{algorithmic}[1]
		\Function{FOIL}{examples}
			\State \textit{Pos} $\gets$ a set of \textbf{positive} examples
			\State \textit{Neg} $\gets$ a set of \textbf{negative} examples
			\State \textit{LearnedRules} $\gets \{ \}$ 
			\While{\textit{Pos}}
				\State \textit{NewRule} $\gets$ LearnRule()
				\State \textit{LearnedRules} $\gets$ \textit{LearnedRules} $+$ \textit{NewRule}
				\State \textit{Pos} $\gets$ \textit{Pos} - \{ covered positive examples \}
			\EndWhile
			\State return \textit{LearnedRules}
		\EndFunction
		\Statex
		\Function{LearnRule}{}
			\State \textit{NewRule} $\gets$ the empty rule that predict the \textit{target} attribute
			\State \textit{NewRuleNeg} $\gets$ \textit{Neg}
			\While{NewRuleNeg}
				\State \textit{CandidateLiterals} $\gets$ generate candidate literals for \textit{NewRule}
				\State \textit{BestLiteral} $\gets  \arg\max_{L \in CandidateLiterals}FoilGain(L, NewRule)$
				\State add \textit{BestLiteral} to precondition of \textit{NewRule}
				\State \textit{NewRuleNeg} $\gets$ subset of \textit{NewRuleNeg} covered by \textit{NewRule}
			\EndWhile		
		\EndFunction
	\end{algorithmic}
\end{algorithm}







\subsection{RIPPER}

The RIPPER algorithm is another instance of the sequential covering algorithms. Compared to FOIL, it introduces two additional steps - rule pruning and rule optimization. The algorithm is summarized in Algorithm \ref{alg:RIPPER}. \\

For multi-class problems, it orders the classes according to increasing class prevalence --  a fraction of instances that belong to a particular class. It learns the rule set for the smallest class first and treats other classes as negative. It then repeats the process taking the next smallest class as a positive class. RIPPER also uses $\mathtt{FoilGain}$ measure as the performance measure.\\

The pruning step is done by using \textit{Incremental Reduced Error Pruning}. The data set is split into the training and validation set. The rule is then induced from the training set. The validation set is used to prune the rules. For each rule, performance measure $v$ is calculated with a precondition removed. The performance measure $v$ is defined as 

\begin{equation}
	v(pruneRule, Pos, Neg) = \frac{p - n}{p + n},
\end{equation}

where $p$ is the number of positive examples in the validation set covered by the pruned rule and $n$ is the number of negative examples in the validation set covered by the pruned rule. \\

In the optimization step, for each rule two new rules are generated. One is completely built from scratch while the second one is generated by adding new conjunctions to the existing rule. As RIPPER performs the greed search, the rule built from scratch might not be the same as the starting one. In that way, three separate rule sets are generated and compared. The resulting rule set is chosen to minimize the \textit{Minimum Description Length} (MDL). MDL uses the assumption that any \textit{regularity} in the data can be used to \textit{compress} the data. Given the hypothesis' set $\mathcal{H}^{(0)}, \mathcal{H}^{(1)}, \ldots, \mathcal{H}^{(n)}$, which in this case are the rule sets, the best hypothesis is the one which minimizes the distance 

\begin{equation}
	L(\mathcal{H}) + L(\mathcal{D} | \mathcal{H}),
\end{equation}

where $L(\mathcal{H})$ is the length, in bits, of the description of the hypothesis and $L(\mathcal{D} | \mathcal{H})$ is the length, in bits, of the description of data when encoded with the help of the hypothesis.


\begin{algorithm}
	\caption{RIPPER}
	\label{alg:RIPPER}
	\begin{algorithmic}[1]
		\Function{RIPPER}{examples, predicates}
			\State \textit{LearnedRules} $\gets \{ \}$
			\State order classes in ascending order according to their prevalence
			\State \textit{CurrentClass} $\gets$ smallest class
			\State \textit{Pos}, \textit{Neg} $\gets$ positive and negative examples of \textit{CurrentClass}
			\While{\textit{Pos}}
				\State split data set into \textit{training} and \textit{validation} set
				\State \textit{NewRule} $\gets$ LearnRule(\textit{training set})
				\State \textit{NewRule} $\gets$ PruneRule(\textit{validation set}, \textit{NewRule})
				\State \textit{LearnedRules} $\gets$ \textit{LearnedRules} + \textit{NewRule}
				\State remove examples covered by \textit{NewRule}
				\State \textit{CurrentClass} $\gets$ smallest class after removing examples
				\State \textit{Pos}, \textit{Neg} $\gets$ positive and negative examples of \textit{CurrentClass}
			\EndWhile
			\State Optimize(\textit{LearnedRules})
			\State return \textit{LearnedRules}
		\EndFunction
		\Statex
		
		\Function{LearnRule}{\textit{Pos}, \textit{Neg}}
			\State \textit{Rule} $\gets \{ \}$
			\While{ there are \textit{Neg} covered}
				\State add conjunct if it improves $\mathtt{FoilGain}$
			\EndWhile
			\State return \textit{Rule}
		\EndFunction
		\Statex
		
		\Function{Prune}{\textit{Pos}, \textit{New},\textit{Rule}}
			\Repeat
				\State \textit{Predicate} $\gets$ select a precondition from \textit{Rule}
				\State $v(pruneRule,Pos,Neg) = \frac{p - n}{p + n}$
				\State delete \textit{Predicate} if it improves $v$ 
			\Until{no deletion improves $v$}
		\EndFunction
		\Statex
		
		\Function{Optimize}{Rules}
			\For{each rule $r$ in \textit{Rules}}
				\State replacement rule $r* \gets$ build new rule
				\State revised rule $r' \gets$ add conjuncts to extend $r$
				\State compare the rule set for $r$ against the rules set for $r*$ and $r'$
				\State choose the rule set that minimizes the MDL
			\EndFor
		\EndFunction
	\end{algorithmic}	

\end{algorithm}






\subsection{Aleph}

\texttt{Aleph} (\textbf{A} \textbf{L}earning \textbf{E}ngine for \textbf{P}roposing \textbf{H}ypotheses) is another commonly used rule induction system. The outline of the approach is summarized in Algorithm \ref{alg:Aleph}.  \\

It starts with a selection of an example to generalize. When the examples is selected, the most specific clause is built. The constructed clause has to entail the selected example and it is usually a \textit{bottom clause} -- a definite clause with many literals. \\

\texttt{Aleph} then performs a search to generalize the bottom clause. Generalization is done by searching for a subset of the literals in the bottom clause that score the best. The search is performed by a \textit{branch-and-bound} search. At this moment \texttt{Aleph} tries to find the best generalization of the current bottom clause, although it does not produce all generalizations. When the best clause is found, it is added to the current theory and all examples covered by the clause are removed. \\

The generalization of a clause is summarized in the function \texttt{GENERALIZE} in Algorithm \ref{alg:Aleph}. It starts by selecting a clause from the \textit{active set}. Clauses are selected in a way that clauses with fewer literals are chosen first. The selected clause is then \textit{branched} by adding one literal at time. For each child, the cost and lower bound are calculated. Lower bounds represent the lower cost that can be obtained at the node (that represent the clause) and the sub-tree below it. 

\begin{algorithm}
	\caption{Aleph}
	\label{alg:Aleph}
	\begin{algorithmic}[1]
		\Function{Aleph}{examples}
			\State \textit{Pos} $\gets$ positive examples
			\State \textit{Neg} $\gets$ negative examples
			\State \textit{theory} $\gets \{ \}$
			\While{\textit{Pos}}
				\State \textit{example} $\gets$ select example
				\State \textit{clause} $\gets \mathtt{BuildMostSpecificClause}$(\textit{example}) 
				\State \textit{rule} $\gets \mathtt{Generalize}$(\textit{clause})
				\State \textit{theory} $\gets$ \textit{theory} $+$ \textit{rule}
				\State Remove redundant
			\EndWhile
		\EndFunction
		\Statex
		\Function{Generalize}{examples,clause}
			\State \textit{activeSet} $\gets$ \{ 0\}
			\State \textit{Cost} $\gets \infty $
			\State \textit{CurrentBest} $\gets$ select random
			\While{\textit{activeSet} contains clauses}
				\State remove first node $k$ from \textit{activeSet}
				\State generate children of $k$, compute costs $Cost_i$ and lower bounds on costs $L_i$
				\For{each child}
					\If{$L_i > C$}
						\State prune child
					\Else						
						\If{child is complete solution \texttt{AND} $Cost_i < C$}
							\State $Cost \gets Cost_i$
							\State \textit{CurrentBest} $\gets$ child
							\State prune clauses in \textit{activeSet} with $L_i$ more than $Cost_i$ 
						\EndIf
						\State add child to \textit{activeSet}
					\EndIf
				\EndFor
			\EndWhile 
		\EndFunction
	\end{algorithmic}
\end{algorithm}


\section{Results and induced rules}

The above described methods are used to extract rules that describe the patterns. Classifiers are evaluated with the 10-fold cross validation. Tables \ref{tab:Precision} and \ref{tab:Recall} summarize the precision and recall of each classifier. Results are represented class-wise together with the number of rules found. \\

When analyzing the performance of the classifier, both accuracy and the number of induced rules should be considered. The results suggest that Decision trees and RIPPER perform the best. The overall accuracy of the decision tree and RIPPER are 95.26 \% and 94.50 \% respectively. The decision tree achieves the highest accuracy, but produces significantly  more rules - 25 compared to 10 induced by RIPPER. Such a difference suggests  overfitting. Induced rules are shown in Appendix \ref{AppendixB}. It can be observed that RIPPER not only produces less rules, but those rules are also much simpler than ones induced by the decision tree. Rules induced by RIPPER usually have one or two clauses in the body, while rules induced by the decision tree usually have three or four clauses in the body. Also, it can be seen that 9 out of 25 rules induced cover less than 10 examples in the dataset. The state-of-the-art performance reported so far is 95.19 \% \cite{Wiliem}. The obtained results are quite comparable to that paper, but also provide explainable results.\\

Taking both performance and simplicity into account, the rules induced by RIPPER seem to perform the best. Their simplicity makes them more applicable to real life scenarios. They also correspond more to the intuition about the patterns. The induced rules mostly extract the most distinguishing property of a certain pattern. For example, \textit{centromere} class is identified by a large number of objects in the cell body, \textit{cytoplasmatic} by the irregular shape and \textit{coarse speckled} by rough texture. All of these properties are characteristic for specific patterns and don't appear for other types. The drawback of this approach is that rules should be always applied \textit{sequentially}. \\

As suggested in the literature, the most difficult classes to classify are \textit{fine speckled} and \textit{homogeneous}. It is very hard to distinguish between those two classes because they look very similar, especially in low intensity images. The literature indicated that mitotic cells are highly informative for this task. It appears that  \texttt{RIPPER} and \texttt{decision trees} are able to capture those relations, while \texttt{ALEPH} and \texttt{FOIL} are not.

\begin{table}
	\caption{Precision of pattern classification}
	\label{tab:Precision}
	\tiny
	\centering
	\begin{tabular}{|c|c|c|c|c|c|c|c|}
		\textbf{method} & homogeneous & nuleolar & centromere & cytoplasmatic & fine speckled & coarse speckled & rules \\
		\hline
		\hline
		RIPPER & 93.48 \% & 94.65 \% & 98.24 \% & 96.33 \% & 83.33 \% & 92.78 \% & 10 \\
		\hline	
		DTREE & 93.48 \% & 97.10 \% & 100 \% & 96.46 \% & 87.00 \% & 96.86 \% & 25 \\	
		\hline
		FOIL & 92.15 \% & 98.59 \% & 98.79 \% & 98.98 \% & 92.84 \% & 92.86 \% & 17 \\ 
	\end{tabular}
\end{table}

\begin{table}
	\caption{Recall of pattern classification}
	\label{tab:Recall}
	\tiny
	\centering
	\begin{tabular}{|c|c|c|c|c|c|c|c|}
		\textbf{method} & homogeneous & nuleolar & centromere & cytoplasmatic & fine speckled & coarse speckled & rules \\
		\hline
		\hline
		RIPPER & 100 \% & 95.44 \% & 93.56 \% & 96.33 \% & 86.54 \% & 85.71 \% & 10 \\
		\hline	
		DTREE & 100 \% & 97.10 \% & 93.56 \% & 100 \% & 93.27 \% & 88.10 \% & 25 \\	
		\hline
		FOIL & 100 \% & 88.78 \% & 92.37 \% & 90.9 \% & 58.57 \% & 84.27 \% & 17 \\ 
	\end{tabular}
\end{table}





\chapter{Conclusion}

\label{Conclusion}

This thesis proposes a solution for computer aided assistance in commonly used diagnostics in autoimmune diseases. The solution closely follows the procedure medical experts suggest. It covers the segmentation task,  the fluorescence intensity level classification and staining pattern classification. The strongest contribution of this thesis is the interpretable approach to prediction models that help medical experts. \\

The work on the segmentation part was motivated by two problems found in previous approaches. Due to the coloring procedure of fluorescence imaging, cells exhibit different properties which makes them hard to detect, especially in noisy images. By finding the background instead of cells directly, that problem was successfully solved with great improvement. Still, a lot of cells overlap.  By using the information about the background, morphological snakes were introduced. Although not always capable of separating overlapping cells, the method demonstrated very promising results. The problematic case were the patterns with bright organelles where those organelles were segmented. \\

The cells were then characterized with \textit{visual concepts} describing their appearance. Deep belief networks were used to map images, or pixel features to \textit{symbolic} ones. Deep approaches to machine learning demonstrated the great potential in symbolic feature recognition. Surprisingly, convolutional components seem not to improve the recognition, but the full potential of this approach is yet to be investigated. \\

The symbolic representations learned in the previous step are then used to mine rules that describe the patterns. Four commonly used rule mining algorithms were compared and proven to perform as accurately as state-of-the-art methods, but offer explanations about their decisions. I believe such an approach is needed when computers assist in making a diagnosis, based on images or other diagnostic tests.

\begin{flushleft}
	\large
	\vspace{15pt}
	\textbf{Future work}
\end{flushleft}

In order to make a fully functional system that supports this process, there are still problems that need to be solved. The segmentation task still suffers from certain problems, such as \textit{overexposed} organelles in a cell's body which end oversegmented. Introducing a model based segmentation method, where movement of the contour is restrained by model learned from data, might be a way to improve the segmentation of cells. \\

There are two actions that have not been mentioned in this thesis, but are of high interest to this task. Mitotic cells were taken as ground truth so far, but detecting and classifying them is a very important task for the diagnostic procedure. Together with mitotic cells, images taken by fluorescence imaging often have artifacts that don't carry any information. Both mitotic cells and artifacts are very rare objects in images which is the main difficulty of this task. Approaching this problem from an unbalanced classification or an outlier detection perspective could be a good start. \\

Symbolic feature learning achieved by deep belief network has proven to have a great potential. As deep learning demonstrates an enormous growth in research, experimenting with different deep approaches to improve symbolic representation learning leaves a lot of space for future work. Special emphasis should be put on methods that can deal with unbalanced classes which usually represent very significant features.
% ... and so on until
%\include{chap-n}
%\include{conclusion}

% If you have appendices:
\appendixpage*          % if wanted
\appendix
%% Chapter Template

\chapter{Analysis of cell's properties} % Main chapter title

\label{AppendixA} % Change X to a consecutive number; for referencing this chapter elsewhere, use \ref{ChapterX}


In this appendix, the distributions of symbolic features are analyzed. The histogram of values for each symbolic features is plotted with regard to class.


%----------------------------------------------------------------------------------------
%	FEATURES
%----------------------------------------------------------------------------------------

\begin{figure}
	\begin{center}
		\includegraphics[width=14cm, height=8cm]{Figures/AppendixA/mitotic_related}
		\caption{The histogram showing how different types of mitotic cells relate to patterns}
	\end{center}
\end{figure}

\begin{figure}
	\begin{center}
		\includegraphics[width=14cm, height=9cm]{Figures/AppendixA/number_of_organelles}
		\caption{The histogram showing how many organelles each pattern express}
	\end{center}
\end{figure}

\begin{figure}
	\begin{center}
		\includegraphics[width=14cm, height=9cm]{Figures/AppendixA/organelle_type}
		\caption{The histogram showing which kind of organelles could be found in certain patterns}
	\end{center}
\end{figure}

\begin{figure}
	\begin{center}
		\includegraphics[width=14cm, height=9cm]{Figures/AppendixA/shape}
		\caption{The histogram showing which shape patterns expose}
	\end{center}
\end{figure}
% Chapter Template

\chapter{Induces rules} % Main chapter title

\label{AppendixB} % Change X to a consecutive number; for referencing this chapter elsewhere, use \ref{ChapterX}


In this appendix, the distributions of symbolic features are analyzed. The histogram of values for each symbolic features is plotted with regard to class.


%----------------------------------------------------------------------------------------
%	FEATURES
%----------------------------------------------------------------------------------------

\begin{figure}
	\caption{Rules induced by RIPPER }
	\label{fig:RulesRIPPER}
	\small
	\centering
	
		\begin{algorithmic}[1]
			\State \texttt{IF} number\_of\_obj = \textit{lots} 
			\Statex \texttt{THEN} \textit{centromere}
			\State \texttt{IF} mitotic\_cells = \textit{bright\_middle} 
			\Statex \texttt{THEN} \textit{homogeneos}
			\State \texttt{IF} organelle\_type = \textit{bright\_on\_dark} 
			\Statex \texttt{THEN} \textit{nucleolar}
			\State \texttt{IF} texture = \textit{rough} 
			\Statex \texttt{THEN} \textit{coarse speckled}
			\State \texttt{IF} shape = \textit{circular}  \texttt{AND} number\_of\_obj = \textit{few}
			\Statex \texttt{THEN} \textit{fine speckled}
			\State \texttt{IF} shape = \textit{irregular} 
			\Statex \texttt{THEN} \textit{cytoplasmatic}
			\State \texttt{IF} mitotic\_cells = \textit{dark spot} \texttt{AND} organelle\_type = \textit{neutral} 
			\Statex \texttt{THEN} \textit{fine speckled}
			\State \texttt{IF} intensity = \textit{intermediate} \texttt{AND} mitotic\_cells = \textit{neutral} 
			\Statex \texttt{THEN} \textit{nucleolar}
			\State \texttt{IF} mitotic\_cells = \textit{neutral} \texttt{AND} speckled = \textit{homogeneous} 
			\Statex \texttt{THEN} \textit{fine speckled}
			\State \textbf{else} \textit{cytoplasmatic}
		\end{algorithmic}
	
\end{figure}


\begin{figure}
	\caption{Rules induced by Decision tree }
	\label{fig:RulesDT}
	\footnotesize
	\centering
	
		\begin{algorithmic}[1]
			\State \texttt{IF} mitotic\_cells = \textit{bright middle} \texttt{AND} number\_of\_obj = \textit{few}
			\Statex \texttt{THEN} \textit{homogeneous}
			
			\State \texttt{IF} mitotic\_cells = \textit{bright middle} \texttt{AND} number\_of\_obj = \textit{lots}
			\Statex \texttt{THEN} \textit{centromere}
			
			\State \texttt{IF} mitotic\_cells = \textit{bright middle} \texttt{AND} number\_of\_obj = \textit{none}
			\Statex \texttt{THEN} \textit{homogeneous}
			
			\State \texttt{IF} mitotic\_cells = \textit{bright middle sparkle}
			\Statex \texttt{THEN} \textit{centromere}
			
			\State \texttt{IF} mitotic\_cells = \textit{dark spot} \texttt{AND} texture = \textit{blob}
			\Statex \texttt{THEN} \textit{cytoplasmatic}
			
			\State \texttt{IF} mitotic\_cells = \textit{dark spot} \texttt{AND} texture = \textit{rough} \texttt{AND} organelle\_type = \textit{bright\_on\_dark}
			\Statex \texttt{THEN} \textit{coarse speckled}
			
			\State \texttt{IF} mitotic\_cells = \textit{dark spot} \texttt{AND} texture = \textit{rough} \texttt{AND} organelle\_type = \textit{dark\_on\_bright} \texttt{AND} speckles = \textit{homogeneous}
			\Statex \texttt{THEN} \textit{fine speckled}
			
			\State \texttt{IF} mitotic\_cells = \textit{dark spot} \texttt{AND} texture = \textit{rough} \texttt{AND} organelle\_type = \textit{dark\_on\_bright} \texttt{AND} speckles = \textit{speckled}
			\Statex \texttt{THEN} \textit{coarse speckled}
			
			\State \texttt{IF} mitotic\_cells = \textit{dark spot} \texttt{AND} texture = \textit{rough} \texttt{AND} organelle\_type = \textit{neutral} 
			\Statex \texttt{THEN} \textit{coarse speckled}
			
			\State \texttt{IF} mitotic\_cells = \textit{dark spot} \texttt{AND} texture = \textit{smooth} \texttt{AND} organelle\_type = \textit{bright\_on\_dark} \texttt{AND} intensity = \textit{intermediate} 
			\Statex \texttt{THEN} \textit{fine speckled}
			
			\State \texttt{IF} mitotic\_cells = \textit{dark spot} \texttt{AND} texture = \textit{smooth} \texttt{AND} organelle\_type = \textit{bright\_on\_dark} \texttt{AND} intensity = \textit{positive} 
			\Statex \texttt{THEN} \textit{nucleolar}
			
			\State \texttt{IF} mitotic\_cells = \textit{dark spot} \texttt{AND} texture = \textit{smooth} \texttt{AND} organelle\_type = \textit{dark\_on\_bright} \texttt{AND} number\_of\_obj = \textit{few} 
			\Statex \texttt{THEN} \textit{fine speckled}
			
			\State \texttt{IF} mitotic\_cells = \textit{dark spot} \texttt{AND} texture = \textit{smooth} \texttt{AND} organelle\_type = \textit{dark\_on\_bright} \texttt{AND} number\_of\_obj = \textit{lots} 
			\Statex \texttt{THEN} \textit{fine speckled}
			
			\State \texttt{IF} mitotic\_cells = \textit{dark spot} \texttt{AND} texture = \textit{smooth} \texttt{AND} organelle\_type = \textit{dark\_on\_bright} \texttt{AND} number\_of\_obj = \textit{none} 
			\Statex \texttt{THEN} \textit{cytoplasmatic}
			
			\State \texttt{IF} mitotic\_cells = \textit{dark spot} \texttt{AND} texture = \textit{smooth} \texttt{AND} organelle\_type = \textit{neutral} 
			\Statex \texttt{THEN} \textit{fine speckled}
			
			\State \texttt{IF} mitotic\_cells = \textit{neutral} \texttt{AND} organelle\_type = \textit{bright\_on\_dark}
			\Statex \texttt{THEN} \textit{nucleolar}
			
			\State \texttt{IF} mitotic\_cells = \textit{neutral} \texttt{AND} organelle\_type = \textit{dark\_on\_bright} \texttt{AND} texture = \textit{rough} \texttt{AND} intensity = \textit{intermediate}
			\Statex \texttt{THEN} \textit{coarse speckled}
			
			\State \texttt{IF} mitotic\_cells = \textit{neutral} \texttt{AND} organelle\_type = \textit{dark\_on\_bright} \texttt{AND} texture = \textit{rough} \texttt{AND} intensity = \textit{positive}
			\Statex \texttt{THEN} \textit{fine speckled}
			
			\State \texttt{IF} mitotic\_cells = \textit{neutral} \texttt{AND} organelle\_type = \textit{dark\_on\_bright} \texttt{AND} texture = \textit{smooth}
			\Statex \texttt{THEN} \textit{fine speckled}
			
			\State \texttt{IF} mitotic\_cells = \textit{neutral} \texttt{AND} organelle\_type = \textit{neutral} \texttt{AND} number\_of\_obj = \textit{few}
			\Statex \texttt{THEN} \textit{nucleolar}
			
			\State \texttt{IF} mitotic\_cells = \textit{neutral} \texttt{AND} organelle\_type = \textit{neutral} \texttt{AND} number\_of\_obj = \textit{lots}
			\Statex \texttt{THEN} \textit{coarse speckled}
			
			\State \texttt{IF} mitotic\_cells = \textit{neutral} \texttt{AND} organelle\_type = \textit{neutral} \texttt{AND} number\_of\_obj = \textit{none} \texttt{AND} intensity = \textit{intermediate}
			\Statex \texttt{THEN} \textit{nucleolar}
			
			\State \texttt{IF} mitotic\_cells = \textit{neutral} \texttt{AND} organelle\_type = \textit{neutral} \texttt{AND} number\_of\_obj = \textit{none} \texttt{AND} intensity = \textit{positive}
			\Statex \texttt{THEN} \textit{fine speckled}
			
			\State \texttt{IF} mitotic\_cells = \textit{unknown} \texttt{AND} number\_of\_obj = \textit{few} 
			\Statex \texttt{THEN} \textit{nucleolar}
			
			\State \texttt{IF} mitotic\_cells = \textit{unknown} \texttt{AND} number\_of\_obj = \textit{none} 
			\Statex \texttt{THEN} \textit{cytoplasmatic}
			
		\end{algorithmic}
\end{figure}


\begin{figure}
	\caption{Rules induced by FOIL }
	\label{fig:RulesFOIL}
	\small
	\centering
	
		\begin{algorithmic}[1]
			\State \texttt{IF} organelle\_type = \textit{bright on dark} \texttt{AND} number\_of\_obj = \textit{lots} 
			\Statex \texttt{THEN} \textit{centromere}
			
			\State \texttt{IF} number\_of\_obj = \textit{lots} \texttt{AND} mitotic\_cells = \textit{bright middle}
			\Statex \texttt{THEN} \textit{centromere}
			
			\State \texttt{IF} intensity = \textit{intermediate} \texttt{AND} speckles = \textit{speckled} \texttt{AND} texture = \textit{rough}
			\Statex \texttt{THEN} \textit{coarse speckled}
			
			\State \texttt{IF} texture = \textit{rough} \texttt{AND} organelle\_type = \textit{neutral} \texttt{AND} mitotc\_cells = \textit{dark spot}
			\Statex \texttt{THEN} \textit{coarse speckled}
			
			\State \texttt{IF} intensity = \textit{positive} \texttt{AND} texture = \textit{rough} \texttt{AND} organelle\_type = \textit{dark on bright} \texttt{AND} number\_of\_obj = \textit{few} \texttt{AND} mitotic\_cells = \textit{dark spot}
			\Statex \texttt{THEN} \textit{coarse speckled}
			
			\State \texttt{IF} shape = \textit{irregular}
			\Statex \texttt{THEN} \textit{cytoplasmatic}
			
			\State \texttt{IF} texture = \textit{blob}
			\Statex \texttt{THEN} \textit{cytoplasmatic}
			
		\end{algorithmic}
	
\end{figure}
% ... and so on until
%\include{app-n}

\backmatter
% The bibliography comes after the appendices.
% You can replace the standard "abbrv" bibliography style by another one.
\bibliographystyle{abbrv}
\bibliography{Bibliography}

\end{document}

%%% Local Variables: 
%%% mode: latex
%%% TeX-master: t
%%% End: 
